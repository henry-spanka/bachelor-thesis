% !TEX root = ../Main.tex
\documentclass[../Main.tex]{subfiles}
\begin{document}

Das de.NBI (Deutsches Netzwerk für Bioinformatik) stellt eine Cloud-Lösung für Bioinformatik an mehreren Standorten in Deutschland zur Verfügung \citep{deNBI}.
Einer dieser Standorte ist am CeBiTec an der Universität Bielefeld angesiedelt.
Im Fokus steht dabei mikrobielle Genetik und Metagenomik. Dabei werden gro{\ss}e Datenmengen
in der Cloud analysiert wie zum Beispiel die Oxford Nanopore Sequenzdaten.

In Bielefeld werden dafür neben standardisierten Instanzen auch sogenannte \glqq high memory\grqq{} Instanzen mit mehr
lokalen Speicherplatz (bis zu 4 TB RAM und 10 TB Festplatte) angeboten\footnote{Freundliche Mitteilung von Jan Krüger (Forschungszentrum Jülich).}. Für hardware-beschleunigte
Instanzen, die Anwendungen im Bereich maschinelles Lernen ausführen, werden spezielle virtuelle Maschinen
mit Grafikkarten (GPUs) bereitgestellt.

Derzeit findet keine Priorisierung der Hardwareanforderungen dieser Instanzen in der de.NBI Cloud statt. Dabei kann es zu einer nicht optimalen
Verteilung der Hardware-Ressourcen kommen. Standardisierte Instanzen werden
auf Hypervisor erstellt auf denen Grafikprozessoren zur Verfügung stehen, wodurch nicht mehr ausreichend CPU-, RAM- und Festplattenressourcen
vorhanden sind um GPU-basierte Instanzen auszuführen.

Ziel dieser Arbeit ist es die Zuweisung der Instanzen an Hypervisor entsprechend zu optimieren, damit möglichst viele Instanzen
ausgeführt werden können.

\biblio % Needed for referencing to working when compiling individual subfiles - Do not remove
\end{document}
