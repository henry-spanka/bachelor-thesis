% !TEX root = ../Main.tex
\documentclass[../Main.tex]{subfiles}
\begin{document}

Wie bereits im vorherigen Kapitel beschrieben bietet sich Kolla zum Deployment
von OpenStack Umgebungen an. Dabei können die einzelnen Dienste modular bereitgestellt werden \citep{KollaAnsible}.

Für die Entwicklung von OpenStack Nova Filter und Weigher werden dabei folgende Dienste benötigt:

\begin{itemize}
    \item Glance
    \item Horizon
    \item Keystone
    \item Neutron
    \item Nova
\end{itemize}

Kolla installiert und konfiguriert in der Standardkonfiguration alle oben genannten notwendigen Dienste.
Darüber hinaus lassen sich alle Konfigurationsdateien anpassen und eigene Pakete installieren welche
für die Erweiterung des Nova Schedulers notwendig sind \citep{KollaAnsible}.

\subsection{Anforderungen}

Kolla setzt auf Ansible Playbooks für das automatische Deployment.

Es werden die folgenden Betriebssysteme unterstützt \citep{KollaSupportMatrix}:

\begin{itemize}
    \item CentOS Stream 8
    \item Debian Bullseye (11)
    \item openEuler 20.03 LTS SP2
    \item RHEL 8 (veraltet)
    \item Rocky Linux 8
    \item Ubuntu Focal (20.04)
\end{itemize}

Daneben bietet Kolla die Abbilder der OpenStack Dienste in mehreren Varianten an \citep{KollaSupportMatrix}:

\begin{itemize}
    \item CentOS
    \item Debian
    \item RHEL
    \item Ubuntu
\end{itemize}

Als Betriebssystem für die Entwicklungsumgebung im Rahmen dieser Bachelorarbeit wurde Ubuntu Focal (20.04) zusammen mit Ubuntu Abbilder
der OpenStack Dienste genutzt. Kolla und die OpenStack Dienste basieren auf dem Yoga Release der am 30. März 2022 veröffentlich wurde \citep{OpenstackYogaRelease}.

Für eine \textbf{all-in-one} Standard-Installation von OpenStack werden mindestens \textbf{2 Netzwerkschnittstellen},
\textbf{8 GB RAM} und \textbf{40 GB Festplattenspeicher} benötigt \citep{KollaQuickStart}.

Zur sinnvollen Erweiterung des Nova Schedulers werden jedoch mindestens zwei Nova Compute Nodes benötigt.
Ein Weighing einer einzelnen Node wäre trivial. Als Entwicklungsumgebung wird daher ein \textbf{multinode}
Deployment bestehend aus einer Controller-, mindestens zwei Compute- und einer Deployment Node verwendet.
Alle Nodes können dabei als Virtuelle Maschinen betrieben werden. Die Anforderungen des Controllers orientieren
sich dabei an denen einer \textbf{all-in-one} Installation. Die Compute Nodes müssen den individuellen Anforderungen
genügen und benötigen ebenfalls zwei Netzwerkschnittstellen.

OpenStack Kolla nutzt dabei eine Schnittstelle für die Verwaltung und Konfiguration der Cloud Dienste.
Die weitere Netzwerkschnittstelle wird von OpenStack Neutron zur Bereitstellung von Netzwerkdiensten der Instanzen benötigt.

Für diese Bachelorarbeit wurden Droplets (Virtuelle Maschinen) vom Anbieter DigitalOcean genutzt. Die Droplets
werden standardgemä{\ss} bereits mit zwei Netzwerkschnittstellen ausgeliefert, weshalb hier keine weitere
Konfiguration notwendig ist.

\biblio % Needed for referencing to working when compiling individual subfiles - Do not remove
\end{document}
