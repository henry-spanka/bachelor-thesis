% !TEX root = ../Main.tex
\documentclass[../Main.tex]{subfiles}
\begin{document}
\label{chapter:PCIPassthrough}

In der de.NBI Cloud werden GPUs über die \textbf{PCI Passthrough} Funktionalität des Hypervisors
in die Instanz durchgereicht. Dabei werden bestimmte Flavors mit Extra Specs versehen um PCI Ressourcen anzufordern:

\begin{minted}[bgcolor=LightGray]
{shell}
openstack flavor set \
--property "pci_passthrough:quadro"="quadro:2" $FLAVOR
\end{minted}

Im obigen Beispiel werden zwei PCI Geräte vom Typ Quadro angefordert. Der Alias muss
dann zusätzlich in der Konfiguration der Nova Compute und API Dienste gesetzt werden \citep{AttachPciToGuests}.
Vendor ID, ProductID und DeviceType sind dabei geräteabhängig und können mit dem Befehl \mintinline{shell}{lspci -nn} ausgelesen werden.
Dabei ist es erforderlich, dass der Gerätetyp auch zu der Passthrough Whitelist des Hosts hinzugefügt wird.

\begin{minted}[bgcolor=LightGray]
{ini}
[pci]
alias = {
    "vendor_id":"10de",
    "product_id":"1cb3",
    "device_type":"type-PCI",
    "name":"quadro"
}
passthrough_whitelist = { "vendor_id": "10de", "product_id": "1cb3" }
\end{minted}

Der von Nova mitgelieferte PCIPassthrough Filter sorgt dafür, dass nur Hosts für
das Scheduling in Betracht gezogen werden, welche auch die angeforderten PCI Ressourcen bereitstellen können \citep{PciPassthroughFilter}.
Eine Gewichtung wird dabei nicht vorgenommen.

\subsection{Scheduling in virtuellen Umgebungen}

Aufgrund der Besonderheit des Scheduling von GPUs, ist es nicht ohne weiteres möglich
dies in einer virtuellen Testumgebung abzubilden. Der Aufbau einer Testumgebung basierend
auf physischer Hardware wäre mit entsprechendem Aufwand und Kosten verbunden, erschwert einen effizienten Aufbau
und lässt kein paralleles Arbeiten zu. Wünschenswert wäre daher eine Implementierung die ein Scheduling
in einer virtuellen Umgebung ermöglicht.

\subsection{Virtueller GPU Filter}
\label{chapter:GPUFilter}

Zur Filterung der Hosts wird ein neuer Filter \textbf{GpuVirtualFilter} implementiert.

\begin{table}[H]
    \centering
    \caption{Attribute des GPU Filters}
    \renewcommand{\arraystretch}{1.2}
    \begin{tabular}{lll}
        \hline
        \textbf{Attribut} & \textbf{Typ} & \textbf{Beschreibung} \\ 
        \hline
        enabled & bool & Aktiviert GPU Filter \\ 
        model & string & GPU Name/Alias \\ 
        count & integer & Anzahl an GPUs \\ 
        \hline
    \end{tabular}
    \label{table:FilterAttributes}
\end{table}

Die neu definierten Attribute in Tabelle \ref{table:FilterAttributes} werden dabei vom Filter unter dem Scope \textbf{gpu\_filter} genutzt.

Jeder Host mit verfügbaren Grafikeinheiten wird in ein Aggregate gruppiert und mit den Metadaten \textbf{model} (GPU Typ) und \textbf{count} (Anzahl vorhandener GPUs) versehen.

Ein Flavor kann GPU Ressourcen anfordern in dem es die Extra Specs setzt:

\begin{enumerate}
    \item \textbf{enabled:} Aktivierung des GPU Filter
    \item \textbf{model:} Angeforderter GPU Typ
    \item \textbf{count:} Anzahl angeforderter GPUs
\end{enumerate}

Der Filter akzeptiert einen Host, wenn mindestens eine Aussage wahr ist:

\begin{enumerate}
    \item Der Filter ist deaktiviert, d.h. Extra Spec \textbf{enabled} != True.
    \item GPU Typ (\textbf{model}) von Flavor und Host sind identisch und
    die Summe von Extra Spec \textbf{count} aller Instanzen und RequestSpec ist kleiner oder gleich der Host Metadata \textbf{count}.
\end{enumerate}

Physische GPUs sind dabei nicht notwendig. Die Filterung wird anhand von Attributen der Flavor und Host Aggregates vorgenommen.
Über die Kombination des Flavors mit den in Kapitel \ref{chapter:PCIPassthrough} genannten Extra Specs, werden dann die virtuellen Ressourcen
an die physische Hardware gebunden.

\subsection{Gleichzeitiges Scheduling mehrerer Instanzen}

Der Nova Scheduler lässt sich unabhängig von einander mehrfach starten. Dies dient zum Beispiel der Lastenverteilung
und Redundanz. Dabei ist nicht garantiert, dass Instanzen sequenziell voneinander scheduled werden \citep{NovaScheduling}.
Dadurch kann es passieren, dass ein Filter einen Host akzeptiert, jedoch vor der Allokation der Instanz ein weiterer Nova Scheduler Prozess den selben
Filter durchläuft und eine Allokation auf dem selben Host vornimmt. Wenn der Filter in einer Entwicklungsumgebung
ohne die Verknüpfung mit PCI Requests (Kapitel \ref{chapter:PCIPassthrough}) ausgeführt wird, führt dies zu
Inkonsistenzen, da keine wirkliche Allokation von Ressourcen stattfindet. Im folgenden wird ein Ansatz vorgestellt, der auf einen virtuellen Filter verzichtet und stattdessen die Nova Placement API
zur Allokation nutzt.

\subsection{GPU Ressourceklasse}
\label{chapter:GPUResourceClass}

Wie bereits in Kapitel \ref{chapter:PlacementAPI} beschrieben, wird die Allokation von Hardwareressourcen vom Nova Placement
Dienst verwaltet. Nova bietet dabei die Möglichkeit der Erweiterung um eigene Ressourceklassen mit dem Präfix \textbf{CUSTOM\_} \citep{NovaCustomResourceClasses}.

Zur Erweiterung der Resource Provider werden standardgemä{\ss}, unter dem Pfad \textbf{/etc/nova/provider\_config/}, Konfigurationsdateien im YAML-Format vom Nova Compute Dienst 
eingelesen \citep{NovaConfigurationOptions}. Die Ressourceklassen werden dann
automatisch dem Placement Dienst mitgeteilt.

Auf den Compute Nodes mit Unterstützung für GPU Ressourcen wird eine neue Ressourceklasse mit dem Namen \textbf{CUSTOM\_GPU} erstellt. Im Beispiel
nehmen wir an, dass maximal 2 GPUs zur Verfügung stehen (total). Das Modell (Quadro) wird dabei über
ein \textbf{Trait} abgebildet. Dies sind Eigenschaften der Ressourceklasse. Die weiteren Parameter sind für das GPU Scheduling nicht notwendig und können
in der OpenStack Dokumentation eingesehen werden \citep{ProviderConfigFile}. Auch hier ist keine Präsenz von
PCI Hardware notwendig. Die Verknüpfung mit GPUs erfolgt, identisch zum GPU Filter (Kapitel \ref{chapter:GPUFilter}), über PCI Requests mit Hilfe von
Extra Specs des Flavors (Kapitel \ref{chapter:PCIPassthrough}).

Ein weiterer Vorteil ist, dass die Nutzung der GPUs nun in einer eigenen Ressourceklasse überwacht wird und über die
Placement API ausgelesen werden kann. Daneben ist die Allokation auch in Testumgebungen atomar und konsistent.

\begin{minted}[bgcolor=LightGray]
{yaml}
# /etc/nova/provider_config/gpu.yaml
meta:
  schema_version: '1.0'
providers:
  - identification:
      uuid: $COMPUTE_NODE
    inventories:
      additional:
        - CUSTOM_GPU:
            total: 2
            reserved: 0
            min_unit: 1
            max_unit: 10
            step_size: 1
            allocation_ratio: 1.0
    traits:
      additional:
        - 'CUSTOM_GPU_QUADRO'
\end{minted}

Die Ressourceklasse kann in den Extra Specs des Flavors angefordert werden.

Dafür wird ein Extra Spec \textbf{resources:\$CUSTOM\_RESOURCE\_CLASS=\$N} zum Flavor hinzugefügt \citep{CustomResourceClassesInExtraSpecs}.
Zur Anforderung von 2 GPUs setzen wir den Extra Spec \textbf{resources:CUSTOM\_GPU=2}.

Die Ressourceklasse unterscheidet nicht zwischen verschiedenen Typen von GPUs. Die Erzwingung eines
bestimmten Typen (hier Quadro) ist mit dem Extra Spec \textbf{trait:CUSTOM\_GPU\_QUADRO=required} möglich \citep{RequestTraitsInNova}.

\subsection{Gewichtung}

In den Kapitel \ref{chapter:GPUFilter} und \ref{chapter:GPUResourceClass} wurden zwei verschiedene Ansätze
zur Filterung von GPU Ressourcen vorgestellt. Dabei wurde bisher keine Gewichtung berücksichtigt.

Dazu wird ein neuer GPU Weigher \textbf{GpuVirtualWeigher} implementiert.

\begin{table}[H]
    \centering
    \caption{Attribute des GPU Weighers}
    \renewcommand{\arraystretch}{1.2}
    \begin{tabular}{lll}
        \hline
        \textbf{Attribut} & \textbf{Typ} & \textbf{Beschreibung} \\ 
        \hline
        enabled & bool & Aktiviert GPU Weigher \\ 
        mode & enum (stack, spread) & Art der Gewichtung \\ 
        resource & string & Resource/Extra Spec der gewichtet werden soll \\ 
        \hline
    \end{tabular}
    \label{table:WeigherAttributes}
\end{table}

Die in Tabelle \ref{table:WeigherAttributes} definierten Attribute werden vom GPU Weigher
unter dem Scope \textbf{gpu\_weigher} genutzt. Ein Flavor, dass gewichtet wird, muss dabei alle Attribute
im Extra Spec konfigurieren.

Ist der Weigher deaktiviert, daher Extra Spec \textbf{enabled} != True, dann hat der Weigher keinen Einfluss auf die Gewichtung ($w = 0$).
Andernfalls summiere den Wert des Extra Spec \textbf{resource} aller Instanzen des Hosts. Dieser gibt
an wie viele GPUs aktuell auf dem Host verwendet werden. Zur Gewichtung der GPU Ressourceklasse wird \textbf{gpu\_weigher:resource=resources:CUSTOM\_GPU}
genutzt.

Über das Attribut \textbf{mode} kann ein Stacking oder Spreading der Instanzen erreicht werden. Beim Stacking
ist dabei das Gewicht identisch zu der Anzahl der genutzten GPUs. Ist ein Spreading gewünscht,
dann wir das Gewicht negiert. Dadurch wird das Gewicht eines Hosts reduziert je mehr GPUs des Hosts bereits
verwendet werden. Dies kann auch genutzt werden um Instanzen ohne GPU Anforderung von GPU-fähigen Hosts fernzuhalten.

\begin{table}[H]
  \centering
  \caption{Beispiel einer Gewichtung}
  \renewcommand{\arraystretch}{1.2}
  \begin{tabular}{|p{3cm}|p{4cm}|c|c|c|c|}
    \hline
    \multirow{2}{3cm}{\textbf{Host}} & \multirow{2}{4cm}{\textbf{Genutzte GPUs}} & \multicolumn{2}{c|}{\textbf{Stacking}} & \multicolumn{2}{c|}{\textbf{Spreading}} \\
    \cline{3-6}
    & & \textbf{$w$} & \textbf{$norm(w)$} & \textbf{$w$} & \textbf{$norm(w)$} \\
    \hline
    compute1 & 3 & 3 & \phantom{$\approx$} 1\phantom{.00} & -3 & \phantom{$\approx$} 0\phantom{.00} \\ \hline
    compute2 & 2 & 2 & $\approx$ 0.66 & -2 & $\approx$ 0.33 \\ \hline
    compute3 & 1 & 1 & $\approx$ 0.33 & -1 & $\approx$ 0.66 \\ \hline
    compute4 & 0 & 0 & \phantom{$\approx$} 0\phantom{.00} & \phantom{-}0 & \phantom{$\approx$} 1\phantom{.00} \\ \hline
  \end{tabular}
  \label{table:WeighingExample}
\end{table}

Durch die Normierung werden die Hosts zwischen 0 und 1 gewichtet. Über den Multiplikator kann das Gewicht
des Weighers verstärkt oder abgeschwächt werden.

\biblio % Needed for referencing to working when compiling individual subfiles - Do not remove
\end{document}
