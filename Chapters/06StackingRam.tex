% !TEX root = ../Main.tex
\documentclass[../Main.tex]{subfiles}
\begin{document}

Zusätzlich zur Gewichtung von GPUs ist ein Stacking von Instanzen auf Basis des Arbeitsspeichers
wünschenswert. Der Nova RAM Weigher verteilt die Instanzen standardgemä{\ss} so breit wie möglich \citep{NormalizedWeights}.
Dies führt bei einer hohen Auslastung der OpenStack Cloud dazu, dass kein oder wenige Hosts Instanzen
mit viel Arbeitsspeicher zur Verfügung stellen können.

Wie in Kapitel \ref{chapter:Weigher} beschrieben wird jeder Weigher mit einem Multiplikator multipliziert. Dieser
lässt sich entweder in der Nova Konfiguration oder über Host Aggregate Metadaten setzen.
Über einen negativen Wert kann dabei ein Stacking anstatt Spreading erzielt werden \citep{ComputeSchedulers}.

\subsection{Kombination von Multiplikatoren}

Die optimale Allokierung von Instanzen auf Hosts lässt sich nicht immer vorhersagen.
Jeder Host kann Arbeitsspeicher bereitstellen, jedoch verfügt nicht jeder Host über GPUs.
Daher betrachten wir GPUs generell als die \glqq wertvollere\grqq{} Ressource.

Über Multiplikator kann der GPU Weigher gegenüber dem RAM Weigher priorisiert werden. Setzen wir
den Multiplikator des GPU Weighers auf $2.0$ und den des RAM Weighers auf $-1.0$, dann wird
die GPU Ressource doppelt gewichtet. In Kombination mit anderen Weigher wie CPU oder Festplattenspeicher
kann dann eine optimale Auslastung erzielt werden.

\biblio % Needed for referencing to working when compiling individual subfiles - Do not remove
\end{document}
